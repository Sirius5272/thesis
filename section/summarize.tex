本章节是对本文提出的方法的总结以及对未来工作的展望。
\section{本文总结}
现代软件开发过程中,开发者常在程序开发领域的知识问答网站如Stack Overflow上讨论与API相关的话题,在API官方文档无法满足程序员的学习需求时,Stack Overflow成为了开发者们获取知识的最佳途径。因此,针对Stack Overflow的API知识挖掘技术应运而生。在之前的研究中,有些学者使用基于规则的方式进行知识挖掘,有些学者则是采用了基于机器学习的方法进行挖掘,均能在特定场景下取得不错的效果,但也都有它们的不足之处。本文针对这一课题提出了新的想法,设计了一种基于自举思想的API描述性知识抽取方法,通过结合目前较为成熟的基于机器学习的文本分类技术和基于规则的知识抽取方法,从Stack Overflow中迭代地抽取出API描述性知识。

现代文本分类技术已经非常成熟且易用,本文使用了Facebook公司开源的fastText文本分类技术,并结合人工制定的规则,对Stack Overflow上的文本句子进行筛选,识别出其中的API描述性知识句子,再在这些句子组成的语料库中进行基于概念变异的API描述性知识实例抽取和基于语言模式变异的API描述性知识元组抽取。对于抽取得到的API描述性知识及其实例,本文还将其构建成一个API描述性知识图谱,并基于知识图谱实现了一个API描述性知识汇总应用。在信息检索方面,图数据库比关系型数据库更为高效、直观。

最后,本文还设计了一系列实验对本方法进行评估,针对本方法的语料库生成模块、抽取主体模块进行了质量评估,并对抽取出来的API描述性知识的有效性以及对API文档的互补性进行了评估。实验结果表明,本文提出的方法抽取得到的API描述性知识具有一定的价值,且能对API文档起到补充作用。
\section{未来展望}
本文提出的方法虽然能从给定的语料库中抽取出API描述性知识,但仍然有缺陷。本文方法可以改进的地方有:
\begin{enumerate}
    \item 本文方法部分依赖于spaCy的性能,虽然本文针对Stack Overflow上的文本特点进行了部分优化,但是依然会有解析错误的情况出现,故未来的一个工作方向可以是针对软件开发相关的自然语言处理继续进行优化,实现更好的解析效果。
    \item 本文只选取了带有Java标签的Stack Overflow帖子作为知识挖掘的语料,而Stack Overflow上还有其他热门语言标签如Python、PHP、JavaScript等,接下来的工作可以专注于提高本方法的泛用性,将API描述性知识的抽取从Java语言拓展到其他语言中。
    \item 本方法进行抽取后构建的API描述性知识图谱结构还比较简单,对API描述性知识的分类也只有三种,而对API知识进行分类这一工作国内外许多学者都有做过相关的研究,所以下一步的工作可以是结合本文方法,对抽取出来的API描述性知识进一步的细分归类,构建更加精细的API描述性知识图谱。
\end{enumerate}